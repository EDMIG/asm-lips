\chapter{Ethical Analysis}


\section{Ethical Justification for the Product}
The primary ethical justification for our product is based on its use with the language learning software for which we are creating it. Ideally, the way our product is used with the language learning application will help people to learn to improve their ability to speak additional languages. This increase in linguistic ability is important because being able to speak multiple languages increases a person's capacity for connecting with others in the world. More people becoming connected in this way fosters a greater sense of global community. Our hope is that the software we are making can help to bring us one step closer to a more socially connected world.


\section{Team and Organizational Ethics}
First and foremost, we place a high priority on being ethical to each other by each performing a fair share of the work. It is unfair for one person to have to do a majority of the project by himself. We intend to clearly plan out which of us will take the lead on which portions of the project so that this will not happen. Should one of us become ill or otherwise indisposed, the other will obviously step in to pick up the slack, but the sick member will do his best to make up for the lost time once he is feeling better.

In addition to our ethical obligations to each other, the two of us also have an ethical obligation to our advisor and the rest of the team working on the full product. We intend to make sure to keep our advisor well-informed of our progress throughout the year. This way she can better aid us if we should run into trouble along the way. We also will not lie about our progress if we fall behind, as this will only hurt us, as well as everyone else involved in the project.

Another ethical issue is plagiarism. It is easy to copy code off the internet and pass it off as one's own without citing the author. We will not do this as it is not only ethically wrong but incredibly easy to avoid. We will simply cite the author of any useful library or snippet of code that we should find and utilize for our lip-tracking code library.

Finally, we will avoid wasting Santa Clara University's resources. Any funding that we should obtain from the school will be appropriately budgeted to make sure none goes to waste and that all of it is beneficial to our project.


\section{Product and Society}
Overall, our project should prove helpful to society by serving to increase people's linguistic diversity. We can, unfortunately, see a nefarious use for our lip-tracking library, which is its use in surveillance applications. It would technically be possible for our project to be used to lip-read people through CCTV cameras and figure out what they are saying. The Criminal Law Handbook states that we should not expect privacy in a public place. Most people, however, expect that their conversations will be private when others are not around them. Despite this, we do not believe that the ethical fault lies within our product, but rather in the way that people might choose to use it. After all, GPU-accelerated lip-tracking algorithms don't spy on people, people spy on people. 

Since our product handles images of people's faces, a concern some people may have is that it might collect and send the photos to a malicious third-party. To combat this, our project's code will also be completely free and open-source. This means that people using our library can feel safe that our code won't do anything undesirable. For example, our code will not collect images or other data of the user and it will not send any data to a remote server. Additionally, open-sourcing the code allows others to benefit from our code by using it for their own projects.
