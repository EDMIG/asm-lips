
\ifethics
\else
\chapter{Ethical Analysis}
\fi

\section{Ethical Justification for the Product}
The world is a big place with many people living in it. Every day, however, advances in technology are increasing the connectivity between people and cultures around the world. This technology provides a means by which anyone around the world could conceivably connect with anyone else. One of the inherent barriers to this process, however, is the fact that not all people speak the same language. In fact, there are thousands of different languages spoken around the world. 

The focus of our Senior Design Project is to develop GPU accelerated lip tracking software that can run reliably and accurately for real-time video processing. We are doing this primarily so that the software can then be used in language learning software that is being developed by the computer engineering and language departments at our university. The language program will use the data obtained with our lip tracking software will provide in addition to audio information from videos of users speaking in front of a camera. This information can then be used to more effectively instruct the user as to how they might improve his or her pronunciation.

The main ethical justification for our product is based on its use with the language learning software for which we are creating it. Ideally, the way our product is used with the language learning application will help people to learn to improve their ability to speak additional languages. This increase in linguistic ability is important because being able to speak multiple languages increases a person's capacity for connecting with others in the world. More people becoming connected in this way fosters a greater sense of global community, which in term can lead to greater global moral awareness. Our hope is that the software we are making can help to bring humankind one step closer to a more socially connected world.

It's use in this language learning software is by no means the only application for our lip-tracking software. There are many other ways that lip-tracking software that is both accurate and fast could be useful. For example, one might use it to design lip reading software that tries to interpret what someone is saying just from seeing his or her mouth move. With such software it would be easier to do automated captioning of video so that blind people might be able to understand what is going on. With translation software this could be expanded to work for  people who do not speak the language of the video as well.


\section{Team and Organizational Ethics}
First and foremost, we place a high priority on being ethical to each other by each performing a fair share of the work. It is unfair for one person to have to do a majority of the project by himself. We intend to clearly plan out which of us will take the lead on which portions of the project so that this will not happen. Should one of us become ill or otherwise indisposed, the other will obviously step in to pick up the slack, but the sick member will do his best to make up for the lost time once he is feeling better.

In addition to our ethical obligations to each other, the two of us also have an ethical obligation to our advisor and the rest of the team working on the full product. We intend to make sure to keep our advisor well-informed of our progress throughout the year. This way she can better aid us if we should run into trouble along the way. We also will not lie about our progress if we fall behind, as this will only hurt us, as well as everyone else involved in the project.

Another ethical issue is plagiarism. It is easy to copy code off the internet and pass it off as one's own without citing the author. We will not do this as it is not only ethically wrong but incredibly easy to avoid. We will simply cite the author of any useful library or snippet of code that we should find and utilize for our lip-tracking code library.

We will also avoid wasting Santa Clara University's resources. Any funding that we should obtain from the school will be appropriately budgeted to make sure none goes to waste and that all of it is beneficial to our project. On top of that, we have minimized the requirements of our project to the point where we may end up not needing any funding at all.

Lastly, all of our actions will comply with the IEEE/ACM code of ethics \cite{acm-ethics}. This means that we will do work in the best interests of our advisor, as long as it is also in the public's best interest. Our code will be written in the highest quality possible and will not be copied unjustly from others. And finally, by making our code open-source, we will allow others to use and learn from our work for their own projects. 


\section{Product and Society}
Overall, our project should prove helpful to society by serving to increase people's linguistic diversity. We can, unfortunately, see a nefarious use for our lip-tracking library, which is its use in surveillance applications. It would technically be possible for our project to be applied in lip-reading software that could be used on people through CCTV cameras and figure out what they are saying. The Criminal Law Handbook states that we should not expect privacy in a public place \cite{criminal-law}. Most people, however, expect that their conversations will be private when others are not around them. Despite this, we do not believe that the ethical fault lies within our product, but rather in the way that people might choose to use it. After all, GPU-accelerated lip-tracking algorithms don't spy on people; people spy on people. 

Since our product handles images of people's faces, a concern some people may have is that it might collect and send the photos to a malicious third-party. To combat this, our project's code will also be completely free and open-source. This means that people using our library can feel safe that our code won't do anything undesirable. For example, our code will not collect images or other data of the user and it will not send any data to a remote server. Additionally, open-sourcing the code allows others to benefit from our code by learning from it and using it for their own projects. 

During beta-testing of our project, we may require having some people test out the software to see how it performs for various faces. We will only store photos of users if they explicitly give us permission to do so and if the photos will help us improve our algorithm. These photos will also not be given out online as a part of the open-source code. We will provide tools that can train the lip-tracker from a collection of photos that the developer will already have procured, but we will not provide any photos ourselves since this may infringe on the privacy of those in the photos. 
