\chapter{Aesthetics Analysis}

\section{Audience and Interface}
Our portion of the project does not have a standard windowed user interface as it is simply a library of code to be used in other applications, such as the language-learning mobile application. For this reason, our aesthetics analysis will focus on how our code is written and presented in documentation. Further, the only probable audience that needs to be considered as far as the aesthetics is concerned is that group of people that would be looking at the code. For this reason, the aesthetics can be targeted entirely toward those with enough technical knowledge to at least partially understand programming and software.


\section{Documentation}
The first is elements of our documentation. It must be easy to read and understand by a variety of audiences. We also create several diagrams within the documentation. These diagrams should be aesthetically pleasing to view as well as easy to understand. For example, lines within flow diagrams should not cross unnecessarily and objects should be arranged in a logical fashion.

Commenting is also a major concern whenever one writes any amount of code. Code becomes exponentially more difficult to understand the more that is written. Comments serve as documentation within the code itself. Proper commenting can help to reduce the amount of time the reader must spend trying to figure out what parts of a body of code do. We will be sure to write descriptive comments, specifying what various portions of our code that might be difficult to ascertain normally do. We won't, however, comment on every little part of our code. Excessive commenting can often lead to just as much confusion as no commenting at all.


\section{Code Simplicity and Clarity}
The second place where aesthetics come into play in our project is in our code. When writing code there are several aesthetic properties to keep in mind. The code should not be overly complicated, which means that it should be as simple and straightforward as possible while still being correct and accomplishing the desired goal. Code should also be split up into useful modules. If code is all written in one place rather than being divided up, it creates monolithic blocks of code that are close to impossible to understand. These considerations lead to elegant and understandable code. Additionally, each module should have logically related functions, providing a high level of cohesion in the overall system. Further, unrelated sections of the code should not reference each other, as it makes it harder for a user to understand what is going on by unnecessarily increasing the coupling of the system.

\section{Code Syntax}
Our code also needs to be written in a consistent and visually appealing way. There are many considerations to make when writing code so that it is visually appealing and easy to read. We will use consistent and proper tabbing. Tabbing assists the reader in following which code falls into what subsection and how it relates to other lines of code. Another consideration is the spacing between variables, operators, and other portions of the code. For example, x = x + 5 is more visually appealing and easier to read than x=x+5. Lastly, adding lines of white space between sections of the code helps the reader to identify where one section of the code stops and another begins.

\section{System Interface}
As our final consideration, will make our library's interface simple and easy to use. The interface of a library of code are those functions which someone intending to use the library would call in order to utilize our code. The goal of our code is to take an image of a face and output a set of points outlining the person's lips. Rather than make the person using our library have to learn how our code works to use it, we will have a function with an input of an image and an output of a list of points. If they want to delve deeper and be able to change some of our settings, we will provide access to those as well, but the core functionality will be very easy to use.
