\chapter{Requirements}

We have several requirements that need to be met in order to implement our desired system. These requirements can be divided into two categories. Functional requirements are those which our system must do. Generally, they can be evaluated only as either true or false, depending on whether or not the system actually does or does not do what it should. Nonfunctional requirements are those which describe the manner in which the functional requirements should be achieved. They are evaluated based on a degree of satisfaction. Additionally, our system also has a couple design constraints, which are limits on the design and implementation of the system. The following requirements are to be met for successful implementation of our lip-tracking library.

\section{Functional}
The system will:
\begin{itemize}
\item Receive camera images from the user.
\item Return a point array of lip edges.
\end{itemize}


\section{Nonfunctional}
The system will be:
\begin{itemize}
\item Efficient - it will run quickly enough for use in a real time system.
\item User Friendly - it will have an easy to use interface.
\item Reliable/accurate - it will produce correct data.
\item Robust - it won't crash. (i.e. Its crash frequency will be within acceptable norms.)
\item Reusable - it will be easy to use in other systems.
\end{itemize}


\section{Design Constraints}
The system must:
\begin{itemize}
\item Be able to run on Android devices and desktop computers.
\item Use CUDA.
\end{itemize}