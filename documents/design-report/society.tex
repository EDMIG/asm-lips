\chapter{Societal Issues}
% See info here: http://www.scu.edu/engineering/cse/ugrad/report.cfm
% We discussed ethical, social, and political in the ethics chapter

\section{Economic}
This lip-tracking library consists of CUDA, OpenCV, and C++ code. Each of these technologies is completely free to use. Additionally, we are making our code free and open-source, so anyone can use it. This puts no economic burden on the creation or use of our lip-tracking library.

\section{Health and Safety}
Our lip-tracking library does not pose any health and safety risks. It is designed to be used while stationary and speaking words in a regular manner. This, by itself, should not cause any injury to the user.

\section{Manufacturability and Usability}
We have designed our system to be very easily modified and rebuilt. We have automated CMake scripts that do the entire build process for the developer using our library, regardless of what operating system they are using. Additionally, our API is simple and straightforward to use, making our library accessible to even novice programmers.

\section{Sustainability}
We based our code on CUDA and OpenCV. These libraries not only continue to be improved, but they both are always backwards-compatible. This means that our library will always work in the future with the latest versions of CUDA and OpenCV. We also used C++ as our main programming language, which will continue to be supported on all the operating systems we are targeting. As a result, our library can either be left alone from now on or continue to be improved over time and it will always be compatible with its applications.

\section{Environmental Impact}
Our lip-tracking library has no perceivable environmental impact. It is designed to be used on the student's mobile device, so no new materials should be used or manufactured to use our system.

\section{Lifelong Learning}
During the creation of this lip-tracking library, we have learned a great deal. We learned the intricacies of CUDA and OpenCV and how to better debug code as well as work on code with improvement in mind rather than development. These lessons will certainly help us once we are working in the industry. Additionally, with the knowledge of these new technologies, we can make better decisions on how to approach solving problems in the future and opened a new programming paradigm to us for future study.

\section{Compassion}
The main goal of this project is to help students learn new languages. There are, however, many more potential uses for this lip-tracking library. For example, it can be used to help the deaf with lip reading. This system definitely has potential to open up many possibilities for reducing the suffering in the world.