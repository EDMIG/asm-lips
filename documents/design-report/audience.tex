\chapter{Audience Analysis}

Our audience will consist of three groups. The first is technical and very knowledgeable of our research area. The second are those who are there to judge us on our writing and oral presentations. The final group will be comprised of those with little to no technical knowledge, including the people who might use the final application that our library runs in as well as any family and friends that might attend to see us present.

\section{Judges - Analytical}
The judges of our presentation are the most important members of our audience, as they are, after all, the ones who inevitably decide how well we do. They would probably have a basic understanding of general engineering concepts, but most likely not any knowledge specific to our project's field. As a result, we would have to explain some concepts to them in a way that they would understand. A combination of concise text and graphics would be useful in this case.


\section{Technical - Analytical/Driver}
The main member of this group that we are expecting in our audience is our project advisor. She will have the strongest technical understanding of the material that we are presenting. Others with her experience and understanding of the subject matter may be in the audience as well, such as our advisor's colleagues or other professors. Members of this audience group would likely understand the subject matter of our presentation as well as we do if not better.


\section{Users and Family/Friends - Expressive/Amiable}
The people using the language learning application that our library powers want it to always work reliably. If it doesn't they would be upset and complain. As for family and friends, they want to support us and try to learn about what we made. Both parties have very little to no knowledge of how our system would work, so explanations to them need to be very simple and straightforward, otherwise, we risk boring or confusing them. Pictures, rather than text are more beneficial here.

\section{Conclusion}
Much of our presentation will be targeted toward the judges, since they have the most impact on our success. We plan to start out general by giving a simple, interface and functionality-based overview of the whole system. This will allow the people who don't want a lot of complicated information to at least understand what our project is about. Once the general information has been covered, we would then go into more information for the judges and other, more technical audience. Given that the judges will be the primary focus of our presentation, we will not spend too much time on the most heavily technical sections so that we do not lose their attention in extensive computation and field specific jargon. 