\chapter{Use Cases}
Based on our conceptual model, our system has a number of actions that users might perform. Any action that a user might take can be considered a use case, which is defined as the steps needed to achieve a goal on the application. Presented below are the use cases for our library, which effectively correspond to the different functions in the library. Under each use case, the actor, goal, precondition, postcondition, sequence of steps, and exceptions are listed. All the use cases of our application are shown below in \Cref{fig:use-cases}. Following the diagram are detailed descriptions of each use case.

\begin{figure}[!h]
	\noindent\centering\resizebox{0.8\textwidth}{!}{
		\begin{tikzpicture}

\begin{umlsystem}[x=4]{Use Cases} 
\umlusecase{~~~~~~~~~~Initialize Tracker~~~~~~~~~~} 
\umlusecase[y=-2]{~~~Get Acceptance Threshold~~~} 
\umlusecase[y=-4]{~~~~Set Acceptance Threshold~~~~}  
\umlusecase[y=-6]{Get Number of Contour Points}
\umlusecase[y=-8]{~~~~~~~~~~Get Lip Contour~~~~~~~~~~}
\umlusecase[y=-10]{~~~~~~~~~~~~Reset Tracker~~~~~~~~~~~~} 
\end{umlsystem} 

\umlactor[y=-5, x=-2.5]{user} 

\umlassoc{user}{usecase-1} 
\umlassoc{user}{usecase-2} 
\umlassoc{user}{usecase-3} 
\umlassoc{user}{usecase-4} 
\umlassoc{user}{usecase-5} 
\umlassoc{user}{usecase-6} 

\end{tikzpicture}
	}
	\caption{Use case diagram}
	\label{fig:use-cases}
\end{figure}

\section{Use Case 1: Initialize Tracker}

\begin{description}
	\item[Actor:] User.
	\item[Goal:] Initialize the tracking program with the shape data used for tracking.
	\item[Preconditions:] User has imported our library (which includes shape data).
	\item[Postconditions:] Tracker is ready to accept images.
	\item[Sequence of Steps:] User calls the \texttt{initializeTracker} with shape data filepath.
\end{description}

\section{Use Case 2: Get Acceptance Threshold}

\begin{description}
	\item[Actor:] User.
	\item[Goal:] Obtain a value for the acceptance threshold for the lip tracking process.
	\item[Preconditions:] User has imported our library.
	\item[Postconditions:] User has acceptance threshold value.
	\item[Sequence of Steps:] User calls the \texttt{getAcceptanceThreshold} function.
	\item[Exceptions:] None.
\end{description}


\section{Use Case 3: Set Acceptance Threshold}

\begin{description}
	\item[Actor:] User.
	\item[Goal:] Change the value for the acceptance threshold for the lip tracking process.
	\item[Preconditions:] User has imported our library.
	\item[Postconditions:] Acceptance threshold value is changed to provided value.
	\item[Sequence of Steps:] User calls the \texttt{setAcceptanceThreshold} function.
\end{description}

\section{Use Case 4: Get Number of Contour Points}

\begin{description}
	\item[Actor:] User.
	\item[Goal:] Obtain the number of contour points provided by the lip-tracking process.
	\item[Preconditions:] User has imported our library.
	\item[Postconditions:] User has the number of contour points.
	\item[Sequence of Steps:] User calls the \texttt{getNumberOfContourPoints} function.
	\item[Exceptions:] None.
\end{description}

\section{Use Case 5: Get Lip Contour}

\begin{description}
	\item[Actor:] User.
	\item[Goal:] Obtain lip contour data from a face image.
	\item[Preconditions:] User has imported our library, has an image to process, and has allocated an array of the proper size.
	\item[Postconditions:] User has a lip contour for the provided image.
	\item[Sequence of Steps:] User calls the \texttt{getLipContour} function on filepath to the facial image.
	\item[Exceptions:] The image provided is not of a face: the function will return an error code.
\end{description}

\section{Use Case 6: Reset Tracker}

\begin{description}
	\item[Actor:] User.
	\item[Goal:] Re-establish accurate tracking of lips.
	\item[Preconditions:] User has imported our library.
	\item[Postconditions:] Tracker is accurately tracking lips in provided images.
	\item[Sequence of Steps:] User calls the \texttt{resetTracker} function.
\end{description}