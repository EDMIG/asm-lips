\chapter{Project Risks}
\Cref{table:risks} below shows the potential risks we have identified for our project. Listed with each risk are the consequences should the risk occur, the probability of the risk occurring, the severity of the risk, the impact of the risk (calculated by multiplying probability and severity), and the mitigation strategies we employed to avoid the risk from occurring.
\newline
\begin{table}[!h]
\def\arraystretch{1.5}
\begin{tabulary}{\textwidth}{|L{2.3cm}|L{2.5cm}|L{2cm}|L{1.5cm}|L{1.3cm}|L{2.9cm}|}
\hline
\textbf{Risk} & \textbf{Consequences} & \textbf{Probability} & \textbf{Severity} & \textbf{Impact} & \textbf{Mitigation} \\
\hline\hline
Poor time management & Not having a finished product. & 0.4 & 7 & 2.8 & Follow gantt chart schedule. Prioritize features of system. \\ \hline
GPU transition problems & Delays. Potentially only CPU implementation finished. & 0.3 & 8 & 2.4 & Consider GPU implementation while working on CPU version. \\ \hline
Team member becomes sick or otherwise disabled & Loss of time. & 0.4 & 4 & 1.6 & Stay informed on each other's progress.Soft deadlines. \\ \hline
Major issues with chosen technologies & Loss of time and progress. & 0.2 & 7 & 1.4 & Early testing of technologies. Consider alternative technologies in advance. \\ \hline
Failure to gather sufficient beta testers & Incomplete testing. & 0.5 & 2 & 1.0 & Scout beta testers in advance. Have thorough alpha testing plan. \\ \hline
%Explosion of the Earth's Sun & Everyone dies. & 1e-27 & 10 & 1e-26 & Pray. Say goodbye to loved one's. \\ \hline
\end{tabulary}
\caption{Risk analysis table}
\label{table:risks}
\end{table}